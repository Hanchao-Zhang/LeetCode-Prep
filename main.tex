\documentclass{assignmeownt}


\coursenumber{LeetCode}
\coursetitle{Solution Manual}
\doctitle{}
\docauthor{Hanchao Zhang}

\begin{document}
\maketitle
\thispagestyle{firststyle}


\tableofcontents
\newpage



\section{Binary Tree}


\question{LeetCode 144, preorder traversal}
\begin{enumerate}
    \item res append val
    \item stack append right
    \item stack append left
\end{enumerate}


\begin{lstlisting}[language=Python, caption=pre-order traversal dfs]
class Solution:
    def preorderTraversal(self, root: Optional[TreeNode]) -> List[int]:
        stack, res = [root], []

        while stack:
            node = stack.pop()
            if node:
                res.append(node.val)
                stack.append(node.right)
                stack.append(node.left)
        
        return res
\end{lstlisting}

\begin{lstlisting}[language=Python, caption=pre-order traversal dp]

class Solution:
    def preorderTraversal(self, root: Optional[TreeNode]) -> List[int]:
        def dfs(node, res):
            if node:
                res.append(node.val)
                dfs(node.left, res)
                dfs(node.right, res)
        
        res = []
        dfs(root, res)
        return res
\end{lstlisting}

\question{LeetCode 94, inorder traversal}

\textbf{iterative solution}
\begin{enumerate}
    \item while cur or stack, stack and keep going to the left node to the last left node
    \item cur = stack.pop and append to the res, then cur goes to the right
\end{enumerate}

\textbf{recursive solution}
\begin{enumerate}
    \item inorder(node.left, res)
    \item res.append val
    \item inorder(node.right, res)
\end{enumerate}

\begin{lstlisting}[caption = inorder traversal iterative]
    class Solution:
    def inorderTraversal(self, root: Optional[TreeNode]) -> List[int]:
        res = []
        stack = []
        cur = root

        while cur or stack:
            while cur:
                stack.append(cur)
                cur = cur.left
            
            cur = stack.pop()
            res.append(cur.val)
            cur = cur.right
        
        return res

\end{lstlisting}


\begin{lstlisting}[language=Python, caption=in-order traversal dp]
class Solution:
    def inorderTraversal(self, root: Optional[TreeNode]) -> List[int]:
        
        def inorder(node, res):
            if node:
                inorder(node.left, res)
                res.append(node.val)
                inorder(node.right, res)
            
        res = []
        inorder(root, res)

        return res

\end{lstlisting}

\question{LeeCode 145, Postorder Traversal}

\textbf{itervative}
\begin{enumerate}
    \item start with root stack, visited False, res empty
    \item while stack, cur = stack.pop, v = visited.pop
    \item if cur, if v: res.append(val)
    \item if cur, if not v: stack cur, visited true, stack cur.right and cur.left and visited both false
\end{enumerate}

\textbf{recursive}
\begin{enumerate}
    \item postorder(node.left, res)
    \item postorder(node.right, res)
    \item res.append(node.val)
\end{enumerate}

\begin{lstlisting}[caption = postorder traversal iterative]
    class Solution:
    def postorderTraversal(self, root: Optional[TreeNode]) -> List[int]:
        stack = [root]
        visited = [False]
        res = []

        while stack:
            cur = stack.pop()
            v = visited.pop()
            if cur:
                if v:
                    res.append(cur.val)
                else:
                    stack.append(cur)
                    visited.append(True)
                    stack.append(cur.right)
                    visited.append(False)
                    stack.append(cur.left)
                    visited.append(False)
        
        return res
\end{lstlisting}

\begin{lstlisting}[caption = postorder traversal dp]
    class Solution:
    def postorderTraversal(self, root: Optional[TreeNode]) -> List[int]:
        res = []
        
        def postorder(node, res):
            if node:
                postorder(node.left, res)
                postorder(node.right, res)
                res.append(node.val)
        
        postorder(root, res)
\end{lstlisting}


\question{LeetCode 102, level order traversal}

\textbf{recursive}
\begin{itemize}
    \item dfs(node, level, levelMap) if not root, return
    \item if level not in levelMap, levelMap[level] = [node.val]
    \item if level in levelMap, levelMap[level].append(node.val)
    \item dfs(node.left, level+1,levelMap), dfs(node.right, level+1, levelMap)
    \item dfs(root, 0, levelMap)
    \item return [x for x in levelMap.keys()]
\end{itemize}

\textbf{iterative}
\begin{itemize}
    \item q with deque([root]), while root
    \item level with empty list, for loop in range(len(q))
    \item node = q.popleft, level append node.val, if left q append left, if right, q append right
    \item result append level
\end{itemize}

\begin{lstlisting}[caption = level ordr traversal dp]
    class Solution:
    def dfs(self, node, level, levelMap):
        if not node:
            return
        
        if level not in levelMap:
            levelMap[level] = [node.val]
        else:
            levelMap[level].append(node.val)
            
        self.dfs(node.left, level + 1, levelMap)
        self.dfs(node.right, level + 1, levelMap)

        
    def levelOrder(self, root: Optional[TreeNode]) -> List[List[int]]:
        levelMap = {}
        self.dfs(root, 0, levelMap)
        return [levelMap[level] for level in sorted(levelMap.keys())]
    
        

\end{lstlisting}


\begin{lstlisting}[caption = level order traversal bfs]
class Solution:
    def levelOrder(self, root: Optional[TreeNode]) -> List[List[int]]:

        if not root:
            return root

        # bfs
        from collections import deque

        # create queue and res for return
        q = deque()
        q.append(root)
        res = []

        while q:
            level = []
            for _ in range(len(q)):
                node = q.popleft()
                level.append(node.val)

                if node.left:
                    q.append(node.left)
                
                if node.right:
                    q.append(node.right)
            res.append(level)
        return res
                    
\end{lstlisting}

\question{LeetCode 104: Maxium Depth}

\textbf{recursive dfs}
\begin{itemize}
    \item if not root, return 0
    \item return max(depth(root.left), depth(root.right)) + 1
\end{itemize}
\textbf{bfs}
\begin{itemize}
    \item general bfs
    \item count the level number 
\end{itemize}
\textbf{iterative dfs}
\begin{itemize}
    \item stack = [[root, 1]]
    \item while stack, pop the depth and root from stack
    \item res = max(res, depth), append([node.left, depth+1]), append([node.right, depth+1])
    \item return res
\end{itemize}

\begin{lstlisting}[caption = dfs recursive]
    class Solution:
    def maxDepth(self, root: Optional[TreeNode]) -> int:
        if not root:
            return 0
            
        return max(self.maxDepth(root.left), self.maxDepth(root.right)) + 1
\end{lstlisting}

\begin{lstlisting}[caption = bfs]
    class Solution:
    def maxDepth(self, root: Optional[TreeNode]) -> int:
        from collections import deque
        if not root:
            return 0
        
        q = deque([root])
        count = 0
        while q:
            for _ in range(len(q)):
                cur = q.popleft()
                if cur.left:
                    q.append(cur.left)
                if cur.right:
                    q.append(cur.right)
            
            count += 1

        return count

\end{lstlisting}

\begin{lstlisting}[caption = dfs iterative]
    class Solution:
    def maxDepth(self, root: Optional[TreeNode]) -> int:
        stack = [[root, 1]]
        res = 0

        while stack:
            node, depth = stack.pop()
            if node:
                res = max(res, depth)
                stack.append([node.left, depth + 1])
                stack.append([node.right, depth + 1])

        return res
\end{lstlisting}



\question{LeetCode 101, symmetric tree}
\textbf{recursive}
\begin{enumerate}
    \item if not node1 and not node2 return True
    \item if only node1 or node2 exist, return false
    \item if node1.val == node2.val, check the node1.left, node2.right and node1.right and node2.left
\end{enumerate}

\textbf{iterative}
\begin{itemize}
    \item split to left and right node
    \item bfs root1 left right, node2 right left, and see if the tree are the same
\end{itemize}

\begin{lstlisting}[caption= recursion]
    class Solution:
    def check_mirror(self, node1, node2):
        if not node1 and not node2:
            return True

        if (node1 and not node2) or (not node1 and node2):
            return False
        
        if node1.val == node2.val:
            return self.check_mirror(node1.left, node2.right) and self.check_mirror(node1.right, node2.left)

    def isSymmetric(self, root: Optional[TreeNode]) -> bool:
        return self.check_mirror(root, root)

\end{lstlisting}

\begin{lstlisting}[caption=iterative]
    class Solution:
    def check_mirror(self, node1, node2):
        if root.left and not root.right:
            return False
        elif root.right and not root.left:
            return False
        elif not root.left and not root.right:
            return True
        leftT,rightT = [root.left.val],[root.right.val]
        q1,q2 = deque(),deque()
        q1.append(root.left)
        q2.append(root.right)
        while q1:
            node = q1.popleft()
            if node.left:
                leftT.append(node.left.val)
                q1.append(node.left)
            if not node.left:
                leftT.append(1000)
            if node.right:
                leftT.append(node.right.val)
                q1.append(node.right)
            if not node.right:
                leftT.append(1000)
        while q2:
            node = q2.popleft()
            if node.right:
                rightT.append(node.right.val)
                q2.append(node.right)
            if not node.right:
                rightT.append(1000)
            if node.left:
                rightT.append(node.left.val)
                q2.append(node.left)
            if not node.left:
                rightT.append(1000)
        return leftT==rightT

\end{lstlisting}

\question{LeetCode 114, path sum}
\textbf{recursive}
\begin{enumerate}
    \item if not root, return False
    \item targetsum - node.val
    \item if the end of leaves, not node.left and node.right check the left and right with targetsum hasPathSum(root.left, targetSum) or hasPathSum(root.right, targetSum)
\end{enumerate}

\begin{lstlisting}[caption=path sum recursion]
    class Solution:
    def hasPathSum(self, root: Optional[TreeNode], targetSum: int) -> bool:
        if not root:
            return False
        
        targetSum -= root.val
        if not root.left and not root.right:
            return targetSum == 0

        return self.hasPathSum(root.left, targetSum) or self.hasPathSum(root.right, targetSum)

\end{lstlisting}


\question{LeetCode 250, count univalue subtree}
\begin{enumerate}
    \item if not node, return True, it is univalue
    \item go dfs(node.left) dfs(node.right), if both of them are false, return false
    \item if the node.left.val exist and does not equal to node.val, not a univalue subtree, return false
    \item if the node.right.val exist and does not equal to node.val, not a univalue subtree, return false
    \item otherwise, count += 1, and return True
\end{enumerate}

\begin{lstlisting}
    class Solution:
    def countUnivalSubtrees(self, root: Optional[TreeNode]) -> int:
        self.count = 0

        def dfs(node):
            if not node:
                return True

            l = dfs(node.left)
            r = dfs(node.right)
            
            if not l or not r:
                return False
            
            if node.left and node.val != node.left.val:
                return False
            
            if node.right and node.val != node.right.val:
                return False
            

            self.count += 1

            return True
        
        dfs(root)

        return self.count
\end{lstlisting}


\question{LeetCode 106, construct binary tree, inorder and postorder}
\begin{enumerate}
    \item Hashmap for inorder, number: index
    \item if l>r return None
    \item root = TreeNode(postorder.pop)
    \item get index of root for inorder indx = Hashmap[root.val]
    \item root.right = helper(idx+1, r)
    \item root.left = helper(l, idx-1)
    \item return root, return helper(0, len(inorder)-1)
\end{enumerate}


\begin{lstlisting}[caption=construct binary tree]
    class Solution:
    def buildTree(self, inorder: List[int], postorder: List[int]) -> Optional[TreeNode]:
        inorderIndx = {v:i for i, v in enumerate(inorder)}
        
        def helper(l, r):

            if l > r:
                return None

            root = TreeNode(postorder.pop())
            idx = inorderIndx[root.val]
            root.right = helper(idx+1, r)
            root.left = helper(l, idx-1)

            return root
        
        return helper(0, len(inorder) - 1)


            
\end{lstlisting}

\question{LeetCode 105, construct binary tree, preorder and inorder}
\begin{enumerate}
    \item HashMap for inorder
    \item if l>r, return None
    \item root preorder pop(0)
    \item find the index of root in inorder
    \item root.right = helper(idx+1, r)
    \item root.left = helper(l, idx-1)
    \item return root, return helper(0, len(inorder) - 1)
\end{enumerate}
\begin{lstlisting}
    class Solution:
    def buildTree(self, preorder: List[int], inorder: List[int]) -> Optional[TreeNode]:
        inorderIndex = {v: i for i, v in enumerate(inorder)}

        def helper(l, r):
            if l > r:
                return None
            
            root = TreeNode(preorder.pop(0))
            idx = inorderIndex[root.val]
            root.left = helper(l, idx-1)
            root.right = helper(idx+1,r)
            return root
        
        return helper(0, len(preorder) - 1)
        
\end{lstlisting}





% \newpage


% \question
% \questionpart{What sounds does a cat make?}
% \par Cats can make a `meow', `purr', or `boioioing' sound.

% \bigskip

% \todo{Add more sounds before the cat overlords punish me.}

% \questionpart{Prove that cats are Gods.}
% We must prove that cats are Gods.

% \begin{proof}
%   \par Let us assume for the sake of contradiction that cats are not Gods.
%   \par This must mean that the world we are living in cannot exist.
%   \par But the world we are living in does exist.
%   \par This must mean that our initial assumption that cats are not Gods must be incorrect.
% \end{proof}

% \par \finalresult{Therefore, cats are Gods.}

% \question{Prove that cats are equal to infinity.}
% \par We must first prove that cats are a number.
% \begin{lemma}
%   Cats are a number.
% \end{lemma}
% \begin{proof}
%   \par Obviously, cats are a number.
% \end{proof}

% \par Since cats are a number, we can now prove that cats are equal to infinity.
% \begin{proof}
%   Let cat be \( x \).
%   \begin{flalign}
%              & x = x - 1  &  & \text{(Duh)} \nonumber \\
%     \implies & x + 1 = x  &  & \nonumber              \\
%     \implies & x = \infty &  & \nonumber
%   \end{flalign}

%   \par We can clearly see that \( x = \infty \).
% \end{proof}

% \par \finalresult{Therefore, cats are equal to infinity.}

\end{document}
